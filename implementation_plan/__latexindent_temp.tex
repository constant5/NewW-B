\documentclass{report}
\usepackage{lmodern}
\usepackage{listings}
\usepackage{color}
\usepackage{fullpage}

\definecolor{dkgreen}{rgb}{0,0.6,0}
\definecolor{gray}{rgb}{0.5,0.5,0.5}
\definecolor{mauve}{rgb}{0.58,0,0.82}

\lstset{frame=tb,
  language=Java,
  aboveskip=3mm,
  belowskip=3mm,
  showstringspaces=false,
  columns=flexible,
  basicstyle={\small\ttfamily},
  numbers=none,
  numberstyle=\tiny\color{gray},
  keywordstyle=\color{blue},
  commentstyle=\color{dkgreen},
  stringstyle=\color{mauve},
  breaklines=true,
  breakatwhitespace=true,
  tabsize=3
}

\begin{document}

Web-based Newsgroup (NewW-B)

\textbf{Description}:\\
We will implement a web based newsgroup platform that will users to browse newsgroup articles. The platform will include a user management
and authentication system, a topic based subscription service, a ranking
and comment system, a browsing interface that prioritizes highly rated
articles, and both simple and advanced search tools.

\emph{Features}

\begin{itemize}
\item
  Download relevant articles

  \begin{itemize}
  \item
    Use RSS feeds to downloads html articles
  \item
    Sort articles into the file system
  \item
    Input article records into relevant DB tables
  \end{itemize}
\item
  Manage User Account

  \begin{itemize}
  \item
    Create and account user profiles
  \item
    Mange usernames and passwords
  \item
    Verify login details
  \end{itemize}
\item
  Subscription service

  \begin{itemize}
  \item
    Subscribe and unsubscribe from news categories
  \end{itemize}
\item
  Browsing Service

  \begin{itemize}
  \item
    Decide what articles and in what order to display to the user
  \item
    Update users view articles history
  \item
    Allow filtering by category, date, etc
  \end{itemize}
\item
  Search

  \begin{itemize}
  \item
    Simple search
  \item
    Advanced search with filtering
  \end{itemize}
\end{itemize}

\textbf{Implementation}

We will construct a NoSQL JSON Document Database with the following
document model:

\newpage
\begin{lstlisting}
const UserSchema =  new Schema({
    u_id : {type: String, required: true},
    f_name : String,
    l_name : String,
    email : String,
    pw : {type: String, required: true},
    create_date : String,
    rank : Number,
    follows : [Tag.schema],
    favorites : [mongoose.Schema.Types.ObjectId],
    voted_on : [
        {article: mongoose.Schema.Types.ObjectId,
         vote : Number}],
    comments : [mongoose.Schema.Types.ObjectId]
});

const ArticleSchema =  new Schema({
    title : String,
    author : String,
    date : String,
    URL : String,
    tags : [Tag.schema],
    rank : Number,
    comments : [Comment.schema]
});

const CommentSchema =  new Schema({
    u_id : {
        type: mongoose.Schema.Types.ObjectId, 
        required: true
    },
    date : String, 
    txt : String,
    rank : Number
});

const TagSchema = new Schema({
    tag:String
});
\end{lstlisting}
\newpage

We selected NoSQL as our DB language because we need the ability to
scale the number of users and the number of articles available to users.
This design also allows inserting new articles and users without the
need to populate all attributes and perhaps to add new attributes as we
expand the functionality of the newsgroup system. We will use MongoDB as
our DBMS running on a server located in our lab.

The front end of our platform will be a web-app created in JavaScript
and using the Node.js framework for dynamically querying the DB and
updating pages.

\textbf{Testing}

\begin{enumerate}
\def\labelenumi{\arabic{enumi}.}
\item
  Mocha unit test

  \begin{enumerate}
  \def\labelenumii{\alph{enumii}.}
  \item
    Method tests:

    \begin{enumerate}
    \def\labelenumiii{\roman{enumiii}.}
    \item
      create\_user\_test(), login\_test(), subscribe\_test(),
      vote\_test(), search\_test, etc
    \end{enumerate}
  \item
    Multi-user test:

    \begin{enumerate}
    \def\labelenumiii{\roman{enumiii}.}
    \item
      Simulate actions of a group of random users interacting with the
      system(i.e. subscribing to categories, reading articles,
      searching, voting, and commenting) while timing the various user
      interactions.
    \end{enumerate}
  \end{enumerate}
\end{enumerate}

\end{document}
